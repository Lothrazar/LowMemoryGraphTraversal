
\documentclass[12pt]{article}


\usepackage{amsfonts,amsthm,amssymb,amsmath}

\usepackage{fancyhdr}

\pagestyle{fancy}
\rhead{\today}
\lhead{ }
%\title{Arithmetic in Game Theory}
\author{Samson Bassett}

\begin{document}
%\maketitle





 \begin{itemize} 
\item \textbf{Rule$\_$Zero}$(v)$: Return $2$. 

\item \textbf{Rule$\_$One}$(p,v)$: If $v$ was entered on port $p$, for $1\le p<d(v)$, return $p+1$.  

\item \textbf{Rule$\_$Two}$(v)$: If $v$ was entered on port $d(v)$, return $d(v)$.

\item \textbf{Rule$\_$Two}$(v)$: If $v$ was entered on port $d(v)+1$, return $1$ 



$G=(V,E)$ is a finite $2$-connected graph.  % finding a Hamiltonian cycle in $G^2$ is trivial.  So $G$ has at least four vertices.  

%\begin{eqnarray*}
%&&T(p,v,\alpha)  = (\textbf{Rule\_Zero}(v),\beta)\\
%&&T(p,v,\beta) = (\textbf{Rule\_One}(p,v),\beta); 1\le p\le d(v)+1\\
%&&T(d(v)+2,v,\beta) = (\textbf{Rule\_Two}(v),\beta)\\
%&&T(d(v)+3,v,\beta) = (\textbf{Rule\_Three}(v),\beta)\\
%\end{eqnarray*}



\begin{center}\begin{tabular*}{\textwidth}{c}\hline\end{tabular*}\end{center}
\noindent \textbf{procedure Port-Numbering}(vertex $v$)
%\begin{center}\begin{tabular*}{0.5\textwidth}{c}\hline\end{tabular*}\end{center}
\begin{algorithmic}[1] 
 

\STATE{Let $a$ be the incoming arc of $v$} %\label{line:t1}
\STATE{Let $b$ be the outgoing arc of $v$}%\label{line:t2}
\STATE{Let $k$ be the number such that $v$ is a relay vertex of $k$ distinct wedges} %\label{line:t3}
\STATE{Let $L_v$ be an empty list of port numbers}
\IF{ $k > 0$} %\label{line:t4}
\STATE{Let $\{W_i\}_{i=1}^k$ be the list of ordered wedges at $v$}% \label{line:t5}
\ENDIF %\label{line:t6}





\IF{$a=b^{-1}$} \label{line:sr1}
%\STATE{$L_v \leftarrow 1$}
\STATE{Assign $d(v)$ to $b$} \label{line:backtrack}
\IF{$k>0$}
\STATE{Assign $[1,\sum_{i=1}^ks_i]$ to $\{W_i\}_{i=1}^k$}\label{line:fill0} %\COMMENT{\emph{ Figures \ref{fig:26}, \ref{fig:36}, \ref{fig:46}  }}
\ENDIF
\ENDIF






\IF{$a\ne b^{-1}$ and both $a$ and $b$ are isolated arcs of $v$}\label{line:sr2}
%\STATE (Figures \ref{fig:11}, \ref{fig:21}, \ref{fig:31}, \ref{fig:41})
\STATE{Assign $1$ to $a^{-1}$}\label{line:iso1}
\STATE{Assign $2$ to $b$} \label{line:iso2}%\COMMENT{\emph{ Figures \ref{fig:11}, \ref{fig:21}, \ref{fig:31}, \ref{fig:41}  }}
\IF{$k>0$}
\STATE{Assign $[3,\sum_{i=1}^ks_i+2]$ to $\{W_i\}_{i=1}^k$}\label{line:fill1}
\ENDIF
\ENDIF

 

\IF{$a\ne b^{-1}$ and both $a$ and $b$ are not isolated arcs of $v$}\label{line:sr3}

   \STATE{Pick $j$ such that $a^{-1}$ is a rib of $W_j$}
   \STATE{Pick $j'$ such that $b^{-1}$ is a rib of $W_{j'}$}







    \STATE{Assign $[1  , s_j]$ to $W_{j}$}\label{line:bothin}

    \STATE{Assign  $[s_{j}+1,s_j+s_{j'}]$ to $W_{j'}$}\label{line:bothout} 
   

    \IF{$k \ge 3$}
      \STATE{Assign $[s_j+ s_{j'}+1 ,\sum_{i\ne j, i\ne j'}s_i]$ to $\{W_i\}_{i\ne j, i\ne j'}$ }\label{line:fill3}
    \ENDIF

%\ENDIF % matches W1 double tied

\ENDIF %end of case

\IF{$a\ne b^{-1}$, $a  $ is an isolated arc of $v$ but $b$ is not}\label{line:sr4}
\STATE{Assign $1$ to $a^{-1}$}\label{line:only1}
\STATE{Pick $j$ such that $b^{-1}$ is a rib of $W_j$}
%Figures \ref{fig:13}, \ref{fig:23}, \ref{fig:33}, \ref{fig:43}

  \STATE{Assign $[2,s_j+1]$ to $W_j$}\label{line:outtied} %\COMMENT{\emph{  Figures \ref{fig:13}, \ref{fig:23}, \ref{fig:33}, \ref{fig:43} }}


\IF{$k \ge 2$}
  \STATE{Assign $[s_j+2,\sum_{i\ne j}s_i+1]$ to $\{W_i\}_{i\ne j}$}\label{line:fill4}
\ENDIF

\ENDIF





\IF{$a\ne b^{-1}$, $b$ is an isolated arc of $v$ but $a$ is not}\label{line:sr5}
\STATE{Assign $2$ to $b$}\label{line:only2}
\STATE{Pick $j$ such that $a^{-1}$ is a rib of $W_j$}
%\STATE (Figures \ref{fig:12}, \ref{fig:22}, \ref{fig:32}, \ref{fig:42})
\STATE{Assign $[d(v)+5-s_j,1]$ to $W_j$}\label{line:intied}% \COMMENT{\emph{  Figures \ref{fig:12}, \ref{fig:22}, \ref{fig:32}, \ref{fig:42}  }}

\IF{$k \ge 2$}
  \STATE{Assign $[s_j+2,\sum_{i\ne j}s_i+1]$ to $\{W_i\}_{i\ne j}$}\label{line:fill5}
\ENDIF
 



 
\ENDIF % matches final case
 
\STATE arbitrary-ports$(v,L_v)$ 

\end{algorithmic} 
\begin{center}\begin{tabular*}{\textwidth}{c}\hline\end{tabular*}\end{center}




\end{document}


